\section{Comparison of scale model with simulation \label{sectComparison}}
Figures \ref{band1}, \ref{band2} and \ref{band3} give a visual representation of the characteristics of the wall on various spatial coordinates. Each figure represents the intensities in a fixed octave band. All intensities are normalized to the maximum intensity of all three octave bands so they can be compared.

The white areas in the experimental graphs represent lacking data (due to a technical error during the measurement).

In figure \ref{band1}, the frequencies under consideration are comparable to the width of the wall. Compared to the high frequency regime of figure \ref{band3}, noticably more intensity can be found behind the barrier due to diffraction. However, this lowest spectral band is only the onset of diffraction, it would have been interesting to examine even lower frequencies, where the wavelength becomes comparable to the \emph{height} of the barrier. Sadly, the data in that frequency regime was not usable due to excessive noise.






\begin{figure}
\figOctaveTwoVariableSpaceNonFig{-1.0cm}{band1}{Intensity (dB) in the 2.5\,kHz to 5\,kHz octave band}
	{bands1}{measured}
	{bandsSim1}{simulated}
\figOctaveTwoVariableSpaceNonFig{-1.0cm}{band2}{Intensity (dB) in the 5\,kHz to 10\,kHz octave band}
	{bands2}{measured}
	{bandsSim2}{simulated}
\figOctaveTwoVariableSpaceNonFig{-1.0cm}{band3}{Intensity (dB) in the 10\,kHz to 20\,kHz octave band}
	{bands3}{measured}
	{bandsSim3}{simulated}
\figOctaveTwoVariableSpaceNonFig{-1.0cm}{band4}{Intensity (dB) in the 20\,kHz to 40\,kHz octave band}
	{bands4}{measured}
	{bandsSim4}{simulated}
\end{figure}




\begin{figure}
\figOctaveTwoNoFigNoCaption{-0.0cm}
	{behind-x1-y1}{$x$ = 13\,cm, $y$ = 6\,cm}
	{behind-x1-y3}{$x$ = 13\,cm, $y$ = 12\,cm}
\figOctaveTwoNoFigNoCaption{-0.0cm}
	{behind-x1-y5}{$x$ = 13\,cm, $y$ = 18\,cm}
	{behind-x1-y7}{$x$ = 13\,cm, $y$ = 24\,cm}
\caption{Results of some measurements close behind the wall, with $x$ the distance to the wall and $y$ the distance above the floor. The curves plotted are the raw measurement without (red) and with (green) correction for the attenuation of high frequencies, and the result of the simulation (black). \label{behindNear}}
\end{figure}

\begin{figure}
\figOctaveTwoNoFigNoCaption{-0.0cm}
	{behind-x10-y1}{$x$ = 40\,cm, $y$ = 6\,cm}
	{behind-x10-y3}{$x$ = 40\,cm, $y$ = 12\,cm}
\figOctaveTwoNoFigNoCaption{-0.0cm}
	{behind-x10-y5}{$x$ = 40\,cm, $y$ = 18\,cm}
	{behind-x10-y7}{$x$ = 40\,cm, $y$ = 24\,cm}
\caption{Results of some measurements far behind the wall, with $x$ the distance to the wall and $y$ the distance above the floor. The curves plotted are the raw measurement without (red) and with (green) correction for the attenuation of high frequencies, and the result of the simulation (black). \label{behindFar}}
\end{figure}





