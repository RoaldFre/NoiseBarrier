\section{Comparison of the scale model with simulation \label{sectComparison}}

\subsection{Detailed results for several measurement positions}
The results of several measurements in front of the wall are shown in figure \ref{beforePlots}. Measurements at a similar position, but behind the wall, are shown in figure \ref{behindPlots}.


\begin{figure}
\figOctaveTwoNoFigNoCaption{-0.0cm}
	{before-x8-y1}{$x$ = 28\,cm, $y$ = 6\,cm}
	{before-x8-y3}{$x$ = 28\,cm, $y$ = 12\,cm}
\figOctaveTwoNoFigNoCaption{-0.0cm}
	{before-x8-y5}{$x$ = 28\,cm, $y$ = 18\,cm}
	{before-x8-y7}{$x$ = 28\,cm, $y$ = 24\,cm}
\caption{Results of a selection of measurements 28\,cm in front of the wall, at different heights. The curves plotted are the raw measurement without (red) and with (green) correction for the attenuation of high frequencies, and the result of the simulation (black). \label{beforePlots}}
\end{figure}

\begin{figure}
\figOctaveTwoNoFigNoCaption{-0.0cm}
	{behind-x6-y1}{$x$ = 28\,cm, $y$ = 6\,cm}
	{behind-x6-y3}{$x$ = 28\,cm, $y$ = 12\,cm}
\figOctaveTwoNoFigNoCaption{-0.0cm}
	{behind-x6-y5}{$x$ = 28\,cm, $y$ = 18\,cm}
	{behind-x6-y7}{$x$ = 28\,cm, $y$ = 24\,cm}
\caption{Results of a selection of measurements 28\,cm behind the wall, at different heights. The curves plotted are the raw measurement without (red) and with (green) correction for the attenuation of high frequencies, and the result of the simulation (black). \label{behindPlots}}
\end{figure}

\begin{figure}
\figOctaveTwoVariableSpaceNonFig{-1.0cm}{band1}{Intensity (dB) in the 2.5\,kHz to 5\,kHz octave band.}
	{bands1}{measured}
	{bandsSim1}{simulated}
\figOctaveTwoVariableSpaceNonFig{-1.0cm}{band2}{Intensity (dB) in the 5\,kHz to 10\,kHz octave band.}
	{bands2}{measured}
	{bandsSim2}{simulated}
\figOctaveTwoVariableSpaceNonFig{-1.0cm}{band3}{Intensity (dB) in the 10\,kHz to 20\,kHz octave band.}
	{bands3}{measured}
	{bandsSim3}{simulated}
\figOctaveTwoVariableSpaceNonFig{-1.0cm}{band4}{Intensity (dB) in the 20\,kHz to 40\,kHz octave band.}
	{bands4}{measured}
	{bandsSim4}{simulated}
\end{figure}

\subsubsection*{Measurements in front of the barrier}

For the measurements in front of the sound barrier, the direct sound that reaches the microphone is windowed out as to only keep signals that have reflected or diffracted from either the wall or the floor.

The spectra in figure \ref{beforePlots} show a typical, broad comb structure due to the small temporal delay in the direct signal and the signal that reaches the microphone by first reflecting off of the floor. This delay increases as the distance increase, creating a comb with more `teeth' the higher the measurement was performed.

The time domain plots of figure \ref{before-x8-y1} and \ref{before-x8-y3} clearly show the first strong reflections from the wall from about 6.5\,ms to 6.8\,ms, followed by some smaller signals that went through multiple reflections and/or diffractions.

The impulse response in figure \ref{before-x8-y7} shows a much weaker initial signal. This is because there is no longer a direct specular reflection from the wall. The signal that reaches the microphone between 6.6 and 6.8\,ms is completely ascribed to diffraction at the top of the wall. The spectrum also shows an attenuation of 20\,dB per decade (whereas the previous spectra are, on average, flat). This is consistent with the rule of thumb of equation \ref{diffractionEq}. It predicts a characteristic frequency $f_0$ of 1.8\,kHz, so there should be diffraction over the entire frequency range of the plot.

Lastly, notice that the spectra and time domain signals follow the general pattern of the simulated data, though the spectra with correction for atmospherical attenuation are consistently higher than the simulated values. This is a possible result of deviating values of the temperature and humidity during the measurements, as these took about 12 hours and were done overnight. Hence the correction may have been too strong. The uncorrected spectra did not exceed those of the simulation.



\subsubsection*{Measurements behind the barrier}

Figure \ref{behindPlots} shows some results of the measurement setup where the measurements were made behind the wall. Here, the corrected spectra match the simulated spectra more intimately.

No time windowing was performed on these measurements, every signal that will get recorded will have passed the `system' we are interested in, i.e. the noise barrier (and floor). 

Figure \ref{behind-x6-y1} represents a low measurement, where we see two sets of peaks one at 4.3 and 4.5\,ms and a second set at 4.6 and 4.8\,ms. Each set is the result of the diffraction over the top of the wall of (1) the direct sound of the source to the top of the wall, and (2) the diffraction of the sound that reached the top of the wall by first reflecting at the floor. The second set of peaks are a diffracted signal that first reflected of off the floor before reaching the microphone. Measurements that were made on higher positions from the floor have these two sets of peaks further apart.


The high frequency regime of the spectrum in figure \ref{behind-x6-y1} shows an attenuation of 20\,dB per decade. This is consistent with the rule of thumb of equation \ref{diffractionEq}, which predicts a characteristic frequency $f_0$ of 1.8\,kHz.

This attenuation due to diffraction can also be found in figures \ref{behind-x6-y3} and \ref{behind-x6-y5}, though in a lesser degree. In figure \ref{behind-x6-y7}, there is a direct path from source to microphone, and this vastly overpowers the small diffraction peaks that are still present (where the diffracted signal reflected of off the floor on the side of the microphone).



\subsection{Global overview of the sound field around the wall}
Figures \ref{band1}, \ref{band2}, \ref{band3} and \ref{band4} give a visual representation of the characteristics of the wall on various spatial coordinates. Each figure represents the intensities in a fixed octave band. All intensities are normalized to the maximum of all considered octave bands so they can be compared.

The white areas in the experimental graphs represent lacking data (due to a technical error during the measurement). Also note that the frequency range in figure \ref{bands1} is bordering on the frequency range where measurement noise becomes significant, hence the rather large deviation from the simulated graph in figure \ref{bandsSim1}

The comb-like shape of the spectrum also prevents from assigning a representative value to the average intensity in the small (absolute, linear) frequency span of those lower octave bands. This, by itself, also explains the noisy appearance of figure \ref{bands1}, even in the simulated part.

Diffraction becomes apparent when comparing \ref{bands4} with \ref{bands3} and \ref{bands2}. Lowering the frequencies by an octave raises the intensities found behind the wall (by approximately 6\,dB).







% vim: spell spelllang=en_us
