\section{Scale model of a noise barrier}
A scale model of a simple noise barrier is investigated to characterize several aspects such as the reflection and diffraction of incident sound.

\subsubsection*{The noise barrier}
The model under consideration is composed of five wooden blocks of dimensions $212\,\mathrm{mm} \times 212\,\mathrm{mm} \times 60\,\mathrm{mm}$. These are positioned side by side to form a barrier that is $1.06\,\mathrm{m}$ long.

This could realistically be a 1:10 scale model of a noise barrier that is 2.12\,m high and 60\,cm thick. 

\subsubsection*{The excitation source}
Working with a 1:10 scale model also means that the wavelengths to be considered should be reduced by a factor of 10.

A suitable means of excitation was found in a small spark source. This produces a signal with sufficiently high frequencies and it is an excellent approximation for an ideal point source.

[foto's, schets]

\subsubsection*{Capturing the sound field}
A programmable $x$-$y$-positioning system was used in order to scan a grid of measurement points in front of and behind the wall. The sound field at each measurement point was recorded with a high frequency microphone connected to an oscilloscope sampling at 5\.MHz.

Because of the very low acoustic energy associated with a single spark, additional pre-amplification was required. Even then, the signal to noise ratio was extremely low. In fact, for the lowest measurement point right behind the wall, the signal to noise ratio was of the order of one in ten!

In order to get a sufficiently clean signal at each point of the grid, each measurement was averaged over a thousand excitations.

[TODO even then significant noise, every 50kHz ... Amplifier?]




XXX Deconvolving to get rid of floor (so pure point source/plane wave approaching) was unsuccessful



% vim: spell spelllang=en_us
