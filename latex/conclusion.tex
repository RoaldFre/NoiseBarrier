\section{Conclusion}
In this paper, three methods to characterise the acoustical properties of a noise barrier were discussed. The conclusions of each are summarized below.

The first approach consists of measuring the acoustical performance of a life sized noise barrier. In order to do this, a program was written in Matlab to send and record sounds and to process the gathered data. The standard method for reflection measurements was assessed and it turned out that the Adrienne method does not account for the directivity of the loudspeaker. In order to overcome this difficulty an alternative method was suggested, which involves placing the speaker higher and further away from the wall. Doing so results in a smaller frequency range, but the effect of the speaker's directivity is diminished. This improved method was used to determine the reflection coefficient of a concrete wall. The results were compared with those of the Adrienne method and it was found that the Adrienne method is subject to the directivity of the speaker, whereas the improved method gave a reflection coefficient of approximately one (as expected for a concrete wall).  Although the improved method might not always be practical, it certainly is more physically correct.

Scalemodel 

Simulation