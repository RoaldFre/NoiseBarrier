\section{Conclusion}
In this paper, three methods to characterize the acoustical properties of a noise barrier were discussed. 

First, measurements on a scale model were explored. Care was taken to correct for spurious attenuation of high frequency components due to absorption from the air. Ways of dealing with the low signal to noise ratio were briefly touched.

These measurements were then compared to a simulation of the exact same setup using a basic time domain finite difference scheme. Noteworthy was the correction to 3D of the 2D simulation and the use of a staggered grid to improve accuracy without sacrificing performance.

The measured results appeared to match with the simulation rather well, at least on a coarse level. A closer look showed some slight deviations, possibly related to atmospheric attenuation or the difference between 2D and 3D for diffuse reflections. Nonetheless, the qualitative properties were very similar in both cases. Diffraction clearly manifested itself, as could be seen in the time domain and in the frequency domain as the characteristic 6\,dB per octane attenuation.

The third and final approach consists of measuring the acoustical performance of a life sized noise barrier. In order to do this, a program was written in Matlab to send and record sounds and to process the gathered data. The standard method for reflection measurements was assessed and it turned out that the Adrienne method does not account for the directivity of the loudspeaker.

In order to overcome this difficulty an alternative method was suggested, which involves placing the speaker higher and further away from the wall. Doing so results in a smaller frequency range, but the effect of the speaker's directivity is diminished. This improved method was used to determine the reflection coefficient of a concrete wall. The results were compared with those of the Adrienne method and it was found that the Adrienne method is subject to the directivity of the speaker, whereas the improved method gave a reflection coefficient of approximately one (as expected for a concrete wall). Although the improved method might not always be practical, it certainly is more physically correct.


% vim: spell spelllang=en_us
