\section{Introduction}
Noise nuisance is, unfortunately, something we encounter a lot in our day to day life. Hence, engineers are doing the best they can to reduce this inconvenience. They design different kinds of noise reducing devices, such as the sound barriers we see next to the highway in all shapes and sizes. 
This project pertains to the physics behind those noise barriers. To be more specific, to the reflection, absorption and diffraction of sound waves at such barriers.

This paper is composed of two main parts: the first part discusses two general ways to characterize the acoustical properties of an object: through the use of a scale model and numerical simulation. Both will be applied on a scale model of a simple noise barrier to XXX


the acoustic characterisation of life sized noise reducing devices. It contains a short description of the method we used to determine the reflection characteristics of a wall; an assessment of the standard measuring method (the Adrienne method) and a comparison of the results that both measurement methods yield at an in situ measurement.

The second part deals with 

But first a theoretical introduction to the relevant acoustical properties.
