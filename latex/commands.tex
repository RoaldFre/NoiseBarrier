\newcommand{\imgdir}{images}

\newcommand{\dif}{\,\mathrm d}

\newcommand{\deriv}[2]{\frac{\mathrm d #1}{\mathrm d #2}}
\newcommand{\derivs}[2]{\mathrm d #1 / \mathrm d #2}

\newcommand{\dt}{\dif t}
\newcommand{\dx}{\dif x}
\newcommand{\dy}{\dif y}

\newcommand{\Dt}{\Delta t}
\newcommand{\Dx}{\Delta x}
\newcommand{\Dy}{\Delta y}

\newcommand{\vecvij}{\vec{v}_{i,j}}
\newcommand{\vxij}{{vx}_{i,j}}
\newcommand{\vyij}{{vy}_{i,j}}
\newcommand{\vij}{v_{i,j}}
\newcommand{\wij}{w_{i,j}}
\newcommand{\pij}{p_{i,j}}

%Loosen up on figure placement restrictions
\renewcommand{\textfraction}{0.05}
\renewcommand{\topfraction}{0.95}
\renewcommand{\bottomfraction}{0.95}
\renewcommand{\floatpagefraction}{0.35}
 

%\fig[htb]{width=width+unit}{name=label}{caption}
\newcommand{\fig}[4][htb]{
    \begin{figure}[#1]
        \begin{center}
	    \includegraphics[#2]{\imgdir/#3}\\
	    %\parbox{#2}{\caption{#4\label{#3}}}
	    %voor: \includegraphics[width=#2]{#3}\\
	    \caption{#4\label{#3}}
        \end{center}
    \end{figure}}


%\figOctave[htb]{name=label}{caption}
\newcommand{\figOctave}[3][htb]{
    \begin{figure}[#1]
	\begin{center}
		\scalebox{0.9}{
			\nonstopmode
			\input{\imgdir/#2.tex}
			\errorstopmode
		}
		\caption{#3\label{#2}}
        \end{center}
    \end{figure}}

% \figuurOctaveTwo[htb]{global label}{global caption}
% 	{name1=label1}{caption1}
% 	{name2=label2}{caption2}
\newcommand{\figOctaveTwo}[7][htb]{
	\begin{figure}[#1]
	\begin{center}
	\hspace{-4cm} %hack to center wider than \textwidth
	\subfigure[#5]{ %sub-caption
	\scalebox{0.9}{
		\nonstopmode
		\input{\imgdir/#4.tex}
		\errorstopmode
		\label{#4}
		\rule[-0.8cm]{0cm}{0cm} %don't overlap with sub-caption
	}
	}
	%
	\rule{0.6cm}{0cm} %don't overlap axis labels
	%
	\subfigure[#7]{
	\scalebox{0.9}{
		\nonstopmode
		\input{\imgdir/#6.tex}
		\errorstopmode
		\label{#6}
		\rule[-0.8cm]{0cm}{0cm}
		}
	}
	\hspace{-4cm} %center hack
	\caption{#3\label{#2}}
	\end{center}
	\end{figure}
}

